
%primo passo: creare un environment box personalizzato
%   Preambolo personalizzato 
%		Inscatolamento di teoremi, dimostrazioni e definizioni con diversi colori
%
%

%%%%%%%%%%%%%%%%%%%%%%%%%%%%%%%%%%%%%%%%%%%%%%%%%%%%%%%%%
% i colori di questa pagina http://en.wikibooks.org/wiki/LaTeX/Colors#The_68_standard_colors_known_to_dvips 
%\usepackage[usenames,dvipsnames]{color}


%%%%%%%%%%%%%%%%%%%%%%%%%%%%%%%%%%%%%%%%%%%%%%%%%%%%%%%%%
%con opzione per il tratteggio
%	\usepackage[framemethod=TikZ]{mdframed}  
	\mdfdefinestyle{dotted}{ tikzsetting={ 
		draw= white, 
		dash pattern = on 3pt off 3pt 
	}	}
   



% 	\usepackage{amssymb} 
   

 

%%%%%%%%%%%%%%%%%%%%%%%%%%%%%%%%%%%%%%%%%%%%%%%%%%%%%%%%%
%definisco l'ambiente theorem
\mdfdefinestyle{theoremstyle}{
	linecolor=blue,
	middlelinewidth=2pt,
	skipabove=10pt,
%	leftmargin=-10pt,
%	rightmargin=-10pt,
	everyline = true,
}

	\newmdtheoremenv[style=theoremstyle]{lemma}{Lemma}[section]
	\newmdtheoremenv[style=theoremstyle]{theorem}{Theorem}[section]
	\newmdtheoremenv[style=theoremstyle]{corollary}{Corollary}[section]
	\newmdtheoremenv[style=theoremstyle]{proposition}{Proposition}[section]


%   Fancy Version

\newcounter{Lemma}[section]
\newenvironment{Lemma}[1][]{%
  \stepcounter{Lemma}%
  \ifstrempty{#1}%
  {\mdfsetup{%
    frametitle={%
      \tikz[baseline=(current bounding box.east),outer sep=0pt]
      \node[line width=1pt,anchor=east,rectangle,draw=blue!20,fill=white]
    {\strut \color{blue}{Lemma}~\theLemma};}}
  }%
  {\mdfsetup{%
    frametitle={%
      \tikz[baseline=(current bounding box.east),outer sep=0pt]
      \node[line width=1pt,anchor=east,rectangle,draw=blue!20,fill=white]
    {\strut \color{blue}{Lemma}~\theLemma:~\color{gray}{#1}};}}%
  }%
  \mdfsetup{innertopmargin=10pt,linecolor=blue!20,%
            linewidth=1pt,topline=true,%
            frametitleaboveskip=\dimexpr-\ht\strutbox\relax,}
  \begin{mdframed}[]\relax%
  }{\end{mdframed}}



%%%%%%%%%%%%%%%%%%%%%%%%%%%%%%%%%%%%%%%%%%%%%%%%%%%%%%%%%
%Ipotesi con barra a sinistra

	\usetikzlibrary{calc}
%	\usepackage{fourier-orns}
	\tikzset{
  		box/.style={
      		rectangle,
      		draw=red,
      		fill=white,
      		scale=1,
      		overlay
    }		}

	\mdfdefinestyle{hypothesis}{%
 		hidealllines=true,leftline=true,
 		skipabove=12,skipbelow=12pt,
 		innertopmargin=0.4em,%
 		innerbottommargin=0.4em,%
 		innerrightmargin=0.7em,%
 		rightmargin=0.7em,%
 		innerleftmargin=1.7em,%
 		leftmargin=0.7em,%
 		middlelinewidth=.2em,%
 		linecolor=blue,%
 		firstextra={\path let \p1=(P), \p2=(O) in ($(\x2,0)+(0,\y1)$) 
                           node[box] {\textcolor{blue}{Hp:}};},%
 		secondextra={\path let \p1=(P), \p2=(O) in ($(\x2,0)+(0,\y1)$) 
                           node[box] {\textcolor{blue}{Hp:}};},%
 		middleextra={\path let \p1=(P), \p2=(O) in ($(\x2,0)+(0,\y1)$) 
                           node[box] {\textcolor{blue}{Hp:}};},%
 		singleextra={\path let \p1=(P), \p2=(O) in ($(\x2,0)+(0,\y1)$) 
                           node[box] {\textcolor{blue}{Hp:}};},%
	}

\newmdenv[style=hypothesis]{hypothesis}




%%%%%%%%%%%%%%%%%%%%%%%%%%%%%%%%%%%%%%%%%%%%%%%%%%%%%%%%%
%Tesi con barra a sinistra

	\mdfdefinestyle{thesis}{%
 		hidealllines=true,leftline=true,
 		skipabove=12,skipbelow=12pt,
 		innertopmargin=0.4em,%
 		innerbottommargin=0.4em,%
 		innerrightmargin=0.7em,%
 		rightmargin=0.7em,%
 		innerleftmargin=1.7em,%
 		leftmargin=0.7em,%
 		middlelinewidth=.2em,%
 		linecolor=blue,%
 		firstextra={\path let \p1=(P), \p2=(O) in ($(\x2,0)+(0,\y1)$) 
                           node[box] {\textcolor{blue}{Th:}};},%
 		secondextra={\path let \p1=(P), \p2=(O) in ($(\x2,0)+(0,\y1)$) 
                           node[box] {\textcolor{blue}{Th:}};},%
 		middleextra={\path let \p1=(P), \p2=(O) in ($(\x2,0)+(0,\y1)$) 
                           node[box] {\textcolor{blue}{Th:}};},%
 		singleextra={\path let \p1=(P), \p2=(O) in ($(\x2,0)+(0,\y1)$) 
                           node[box] {\textcolor{blue}{Th:}};},%
	}

\newmdenv[style=thesis]{thesis}


%%%%%%%%%%%%%%%%%%%%%%%%%%%%%%%%%%%%%%%%%%%%%%%%%%%%%%%%%
%Dimostrazione, Tratteggiato da appendere sotto la proposizione
\mdfdefinestyle{proofstyle}{
	style = dotted,
	linecolor=blue,
	skipabove=0pt,
	topline =false,
}
 
 
    \newenvironment{proof}%
    {%
    \begin{mdframed}[style = proofstyle,frametitle={Proof:}]
    }
    {%
    	\begin{flushright}
			$\square$
		\end{flushright}
    \end{mdframed}%
    }%

%%%%%%%%%%%%%%%%%%%%%%%%%%%%%%%%%%%%%%%%%%%%%%%%%%%%%%%%%
%DEFINIZIONE
\mdfdefinestyle{definitionstyle}{
	linecolor=red,
	middlelinewidth=2pt,
	frametitlerule=true,
	frametitlerulecolor=green!60,
%	frametitlerulewidth =1pt,
%	innertopmargin= \topskip ,
	skipabove=3pt,
}

\mdtheorem[ style = definitionstyle ]{definition}{Definition}

%%%%%%%%%%%%%%%%%%%%%%%%%%%%%%%%%%%%%%%%%%%%%%%%%%%%%%%%%
%NOTAZIONE   (WIP)
\mdfdefinestyle{notationstyle}{
	roundcorner=5pt,
	linecolor=green,
	middlelinewidth=2pt,
	frametitlerule=true,
	frametitlerulecolor=green!60,
%	frametitlerulewidth =1pt,
%	innertopmargin= \topskip ,
	frametitle={Notation fixing},
	skipabove=3pt,
}

\newmdenv[style=notationstyle]{notationfix}





%%%%%%%%%%%%%%%%%%%%%%%%%%%%%%%%%%%%%%%%%%%%%%%%%%%%%%%%%
%TAKE AWAY MASSAGE (WIP)
\mdfdefinestyle{takeawaystyle}{
	linecolor=gray,
	middlelinewidth=1pt,
	frametitlerule=true,
	frametitle={Take Away Message},
	skipabove=3pt,
}

\newmdenv[style=takeawaystyle]{TAM}





%%%%%%%%%%%%%%%%%%%%%%%%%%%%%%%%%%%%%%%%%%%%%%%%%%%%%%%%%
% EXAMPLE  (WIP)
\mdfdefinestyle{examplestyle}{
  	linecolor=yellow,
   % frametitle={\textbf{Example:}\colorbox{white}{\space#1\space}},
    innertopmargin=10pt,
    frametitleaboveskip=-\ht\strutbox,
    frametitlealignment=\center
}

\mdtheorem[style=examplestyle]{example}{Example: }

% Fancy http://tex.stackexchange.com/questions/69148/how-to-insert-title-in-mdframed





%%%%%%%%%%%%%%%%%%%%%%%%%%%%%%%%%%%%%%%%%%%%%%%%%%%%%%%%%
% Observation (WIP)
\mdfdefinestyle{observationstyle}{
	linecolor=magenta,
	middlelinewidth=1pt,
	frametitlerule=true,
%	frametitle={Observation},	
	skipabove=3pt,
}

%\newmdenv[style=observationstyle]{observation}
\mdtheorem[ style = observationstyle ]{observation}{Observation}




%%%%%%%%%%%%%%%%%%%%%%%%%%%%%%%%%%%%%%%%%%%%%%%%%%%%%%%%%
% Remark (WIP)
\mdfdefinestyle{remarkstyle}{
	linecolor=cyan,
	middlelinewidth=1pt,
	frametitlerule=true,
	frametitle={Remark:},	
	skipabove=3pt,
}

\newmdenv[style=remarkstyle]{remark}

%%%%%%%%%%%%%%%%%%%%%%%%%%%%%%%%%%%%%%%%%%%%%%%%%%%%%%%%%
% NotaBene (WIP)
\newenvironment{NB}[1]
    {
		\begin{description}
			\item[N.B. :]
			
			
    }
    { 
		\end{description}    
	}
