\documentclass[a4paper,12pt]{scrartcl}

%Dummy text
\usepackage{lipsum}



\usepackage{theoremtemplate}

\begin{document}

	\begin{definition}[Title]
		\lipsum[2]
	\end{definition}

	\begin{lemma}[Title]
		\lipsum[2]
	\end{lemma}
	\begin{proposition}[Title]
		\lipsum[2]
	\end{proposition}
	\begin{theorem}[Title]
		\lipsum[2]
	\end{theorem}
	\begin{corollary}[Title]
		\lipsum[2]
	\end{corollary}

	\begin{Lemma}[Distribution of inner products]\label{thm:stokesTheoremGeometricAlgebraII:1420}
		Given two blades $A_s, B_r$ with grades subject to $0 < r < s$, and a vector $b$, the inner product distributes according to
		\begin{equation}
    		A_s \cdot \left( b \wedge B_r \right) = \left( A_s \cdot b \right) \cdot B_r.
		\end{equation}

		\begin{hypothesis}
			f tale che
				\begin{enumerate}
					\item 
				\end{enumerate}
		\end{hypothesis}
		
		\begin{thesis}
			bubusettete
				\begin{enumerate}
					\item f bulla
				\end{enumerate}
		\end{thesis}

	\end{Lemma}

	\begin{proof}
		ciao
	\end{proof}

	QUI FUNZIONA MALE!
	\begin{proposition}[Fiber Bundle as a Categoty.]
		\begin{hypothesis}
				\begin{itemize}
					\item $\mathfrak{C} = $ set of all possible fiber bundle.
					\item $hom(\mathfrak{C}) = $ set of all bundle morphism.
				\end{itemize}
		\end{hypothesis}
		\begin{thesis}
					The couple ( C, hom(C)) form a concrete category.
		\end{thesis}
	\end{proposition}
	\begin{proof}
		ciao
	\end{proof}

	\begin{notation}
		\lipsum[2]
	\end{notation}
	
	\begin{TAM}
		\lipsum[2]
	\end{TAM}
	
	\begin{example}
		\lipsum[2]
	\end{example}
	
	\begin{observation}
		\lipsum[2]
	\end{observation}
	
	\begin{remark}
		\lipsum[2]
	\end{remark}


\section{List of all the sentences}

\subsection{Sentence Matematica}

\begin{itemize}
 \item Preambolo
 \item Definizione
 \item Osservazione
 \item Lemma
 \item Teorema
 \item Proposizione
 \item Corollario
 \item Fissare la notazione:
 	\\ Quando si da un nome particolare ad una piccola sottodefinizione oppure quando si attribuisce un simbolo a qualcosa
 \item Esempio 	
 \end{itemize}

\subsection{Sentence Fisica}
\begin{itemize}
 \item assioma di matematizzazione 
 	\\(e.g. spazio-tempo = varietà 4D)
 \item assioma come concetto primitivo 
 	\\(e.g. lo spazio fisico ha 3 dimensioni)
 \item assioma di reinterpretazione 
 	\\(e.g. non mi viene in mente ora... è qualcosa che dopo tanti calcoli tiro fuori e gli appioppo una qualche interpretazione fisica)
 \item giustificazione assioma 
 	\\L'assioma è qualcosa che non si può dimostrare ma tendenzialmente si può dare un argomento che cerca di giustificare il fatto che viene usato.. eg. l'omogeneità nello spazio-tempo presa per vera nei modelli FRW.
 \item Conferma sperimentale
 	\\ si mette qua dentro la reference a qualche espiremento che avvalora gli assiomi o altro
 \item Informazioni sul contesto.
 	\\ esempio, studio proteine, in questo contesto gli ordini di grandezza sono ...
 \item Postulato
 	\\Ci sono differenze tra postulato e assioma... assioma= verità autoevidente, postulato= affermazione assunta vera


\end{itemize}




\end{document}


























\begin{document}


\section{Approvati (MD THEOREM di base)}
\lipsum[1]

\begin{lemma}[di coso]
	Enunciato...
Text Text Text Text Text Text $\sqrt{r}$ Text Text Text Text Text

	\begin{hypothesis}
		f tale che
			\begin{enumerate}
				\item 
			\end{enumerate}
	\end{hypothesis}
	
	\begin{thesis}
		bubusettete
			\begin{enumerate}
				\item f bulla
			\end{enumerate}
	\end{thesis}

\end{lemma}%

\begin{theorem}%
teorema
\end{theorem}%

\begin{corollary}%
corollary
\end{corollary}%

\begin{proposition}%
proposition
\end{proposition}%

\begin{Lemma}[Distribution of inner products]\label{thm:stokesTheoremGeometricAlgebraII:1420}
Given two blades $A_s, B_r$ with grades subject to $0 < r < s$, and a vector $b$, the inner product distributes according to
\begin{equation}
    A_s \cdot \left( b \wedge B_r \right) = \left( A_s \cdot b \right) \cdot B_r.
\end{equation}

	\begin{hypothesis}
		f tale che
			\begin{enumerate}
				\item 
			\end{enumerate}
	\end{hypothesis}
	
	\begin{thesis}
		bubusettete
			\begin{enumerate}
				\item f bulla
			\end{enumerate}
	\end{thesis}

\end{Lemma}

\begin{proof}
ciao
\end{proof}

QUI FUNZIONA MALE!
\begin{proposition}[Fiber Bundle as a Categoty.]
	\begin{hypothesis}
			\begin{itemize}
				\item $\mathfrak{C} = $ set of all possible fiber bundle.
				\item $hom(\mathfrak{C}) = $ set of all bundle morphism.
			\end{itemize}
	\end{hypothesis}
	\begin{thesis}
				The couple ( C, hom(C)) form a concrete category.
	\end{thesis}
\end{proposition}


\begin{definition}[Bu]
bbbb bbbbbb bbbbb bbbbb bbbbbbbbb bbbbb bbbb bbbbb bbbbbbbb bbbbbb bbbbbbbbbb bbbb bbbbb bbbbbb 
bbbbbbbb
\end{definition}


\section{WiP}

\begin{notationfix}
Text'
\end{notationfix}

\begin{TAM}
Text
\end{TAM}

\begin{example}
Text
\end{example}

\begin{observation}
Text
\end{observation}

\begin{remark}
Text
\end{remark}


\section{List of all the sentences}

\subsection{Sentence Matematica}

\begin{itemize}
 \item Preambolo
 \item Definizione
 \item Osservazione
 \item Lemma
 \item Teorema
 \item Proposizione
 \item Corollario
 \item Fissare la notazione:
 	\\ Quando si da un nome particolare ad una piccola sottodefinizione oppure quando si attribuisce un simbolo a qualcosa
 \item Esempio 	
 \end{itemize}

\subsection{Sentence Fisica}
\begin{itemize}
 \item assioma di matematizzazione 
 	\\(e.g. spazio-tempo = varietà 4D)
 \item assioma come concetto primitivo 
 	\\(e.g. lo spazio fisico ha 3 dimensioni)
 \item assioma di reinterpretazione 
 	\\(e.g. non mi viene in mente ora... è qualcosa che dopo tanti calcoli tiro fuori e gli appioppo una qualche interpretazione fisica)
 \item giustificazione assioma 
 	\\L'assioma è qualcosa che non si può dimostrare ma tendenzialmente si può dare un argomento che cerca di giustificare il fatto che viene usato.. eg. l'omogeneità nello spazio-tempo presa per vera nei modelli FRW.
 \item Conferma sperimentale
 	\\ si mette qua dentro la reference a qualche espiremento che avvalora gli assiomi o altro
 \item Informazioni sul contesto.
 	\\ esempio, studio proteine, in questo contesto gli ordini di grandezza sono ...
 \item Postulato
 	\\Ci sono differenze tra postulato e assioma... assioma= verità autoevidente, postulato= affermazione assunta vera


\end{itemize}


\end{document}