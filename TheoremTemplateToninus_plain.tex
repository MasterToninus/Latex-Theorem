\usepackage{amsthm}


%%%%%%%%%%%%%%%%%%%%%%%%%%%%%%%%5
%	Riferimento:
%	http://www.dm.unibo.it/studenti/latex/tesi.php
%%%%%%%%%%%%%%%%%%%%%%%%%%%%%%%%5





%%%%%%%%%%%%%%%%%%%%%%%%%%%%%%%%%%%%%%%%%%%%%%%%%%%%%%%%%%%%%%%%%%%%%%%%%%%%%%%%%%%%%%%%%%%%%%%%%%%%%%%%%%%%%%%%%%%%%%%%%%%%%%%%
% TEOREMI

\theoremstyle{plain}
\newtheorem{theorem}{Theorem}[section]
\newtheorem{corollary}[thm]{Corollary}
\newtheorem{lemma}[thm]{Lemma}
\newtheorem{proposition}[thm]{Proposition}






\theoremstyle{remark}
\newtheorem{oss}{Osservazione}







%%%%%%%%%%%%%%%%%%%%%%%%%%%%%%%%%%%%%%%%%%%%%%%%%%%%%%%%%
% Dimostrazione, Tratteggiato da appendere sotto la proposizione
\newenvironment{proof}[1]
    {
		\proof
    }
    { 
		\endproof
	}




%%%%%%%%%%%%%%%%%%%%%%%%%%%%%%%%%%%%%%%%%%%%%%%%%%%%%%%%%
%DEFINIZIONE
\theoremstyle{definition}
\newtheorem{definition}{Definition}[chapter]

%%%%%%%%%%%%%%%%%%%%%%%%%%%%%%%%%%%%%%%%%%%%%%%%%%%%%%%%%
%NOTAZIONE   (WIP) (troppo non standard, nel plain deve diventare definizione base o testo semplice)
\newtheorem{notationfix}{Definition}[chapter]





%%%%%%%%%%%%%%%%%%%%%%%%%%%%%%%%%%%%%%%%%%%%%%%%%%%%%%%%%
% Remark (WIP)
\theoremstyle{remark}
\newtheorem{remark}{Remark}

%%%%%%%%%%%%%%%%%%%%%%%%%%%%%%%%%%%%%%%%%%%%%%%%%%%%%%%%%
%TAKE AWAY MASSAGE (WIP)
\newtheorem{TAM}{Remark}

%%%%%%%%%%%%%%%%%%%%%%%%%%%%%%%%%%%%%%%%%%%%%%%%%%%%%%%%%
% Observation (WIP)
\newtheorem{observation}{Remark}



%%%%%%%%%%%%%%%%%%%%%%%%%%%%%%%%%%%%%%%%%%%%%%%%%%%%%%%%%
% EXAMPLE  (WIP)
\mdfdefinestyle{examplestyle}{
  	linecolor=yellow,
   % frametitle={\textbf{Example:}\colorbox{white}{\space#1\space}},
    innertopmargin=10pt,
    frametitleaboveskip=-\ht\strutbox,
    frametitlealignment=\center
}

\mdtheorem[style=examplestyle]{example}{Example: }

% Fancy http://tex.stackexchange.com/questions/69148/how-to-insert-title-in-mdframed







%%%%%%%%%%%%%%%%%%%%%%%%%%%%%%%%%%%%%%%%%%%%%%%%%%%%%%%%%
% NotaBene (WIP)
\newenvironment{NB}[1]
    {
		\begin{description}
			\item[N.B. :]
    }
    { 
		\end{description}    
	}

%%%%%%%%%%%%%%%%%%%%%%%%%%%%%%%%%%%%%%%%%%%%%%%%%%%%%%%%%
% Danger box (WIP).
% It encapsulates a piece of text that needs revision
% http://tex.stackexchange.com/questions/52023/mdframed-put-something-on-the-start-of-one-vertical-left-rule

\definecolor{warningColor}{named}{red}
\tikzset{
  warningsymbol/.style={
      rectangle,
      draw=warningColor,
      fill=white,
      scale=1,
      overlay}
}

\mdfdefinestyle{warning}{%
 hidealllines=true,leftline=true,
 skipabove=12,skipbelow=12pt,
 innertopmargin=0.4em,%
 innerbottommargin=0.4em,%
 innerrightmargin=0.7em,%
 rightmargin=0.7em,%
 innerleftmargin=1.7em,%
 leftmargin=0.7em,%
 middlelinewidth=.2em,%
 linecolor=warningColor,%
 fontcolor=warningColor,%
 firstextra={\path let \p1=(P), \p2=(O) in ($(\x2,0)+0.5*(0,\y1)$) 
                           node[warningsymbol] {\danger};},%
 secondextra={\path let \p1=(P), \p2=(O) in ($(\x2,0)+0.5*(0,\y1)$) 
                           node[warningsymbol] {\danger};},%
 middleextra={\path let \p1=(P), \p2=(O) in ($(\x2,0)+0.5*(0,\y1)$) 
                           node[warningsymbol] {\danger};},%
 singleextra={\path let \p1=(P), \p2=(O) in ($(\x2,0)+0.5*(0,\y1)$) 
                           node[warningsymbol] {\danger};},%
}

\newmdenv[style=warning]{Warning}
