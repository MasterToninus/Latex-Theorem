\documentclass[a4paper,12pt]{report} 

%Dummy text
\usepackage{lipsum}



\usepackage{amsthm}


%%%%%%%%%%%%%%%%%%%%%%%%%%%%%%%%5
%	Riferimento:
%	http://www.dm.unibo.it/studenti/latex/tesi.php
% http://tex.stackexchange.com/questions/45817/theorem-definition-lemma-problem-numbering
%%%%%%%%%%%%%%%%%%%%%%%%%%%%%%%%5





%%%%%%%%%%%%%%%%%%%%%%%%%%%%%%%%%%%%%%%%%%%%%%%%%%%%%%%%%%%%%%%%%%%%%%%%%%%%%%%%%%%%%%%%%%%%%%%%%%%%%%%%%%%%%%%%%%%%%%%%%%%%%%%%
% TEOREMI

\theoremstyle{plain}
\newtheorem{theorem}{Theorem}[section]
\newtheorem{corollary}[theorem]{Corollary}
\newtheorem{lemma}[theorem]{Lemma}
\newtheorem{proposition}[theorem]{Proposition}







%%%%%%%%%%%%%%%%%%%%%%%%%%%%%%%%%%%%%%%%%%%%%%%%%%%%%%%%%
%DEFINIZIONE
\theoremstyle{definition}
\newtheorem{definition}{Definition}[chapter]


%%%%%%%%%%%%%%%%%%%%%%%%%%%%%%%%%%%%%%%%%%%%%%%%%%%%%%%%%
%NOTAZIONE   (WIP) (troppo non standard, nel plain deve diventare definizione base o testo semplice)
%\newtheorem{notationfix}{Definition}[chapter]
\newenvironment{notationfix}[1]
    {
		\begin{definition}
		
    }
    { 
    
		\end{definition}
	}




%%%%%%%%%%%%%%%%%%%%%%%%%%%%%%%%%%%%%%%%%%%%%%%%%%%%%%%%%
% Remark (WIP)
\theoremstyle{remark}
\newtheorem{remark}{Remark}

%%%%%%%%%%%%%%%%%%%%%%%%%%%%%%%%%%%%%%%%%%%%%%%%%%%%%%%%%
%TAKE AWAY MASSAGE (WIP)
%\newtheorem{TAM}{Remark}
\newenvironment{TAM}[1]
    {
		\begin{remark}
		
    }
    { 
    
		\end{remark}
	}


%%%%%%%%%%%%%%%%%%%%%%%%%%%%%%%%%%%%%%%%%%%%%%%%%%%%%%%%%
% Observation (WIP)
%\newtheorem{observation}{Remark}
\newenvironment{observation}[1]
    {
		\begin{remark}
    }
    { 
		\end{remark}
	}


%%%%%%%%%%%%%%%%%%%%%%%%%%%%%%%%%%%%%%%%%%%%%%%%%%%%%%%%%
% EXAMPLE  (WIP)
\newtheorem{example}{Example}



%%%%%%%%%%%%%%%%%%%%%%%%%%%%%%%%%%%%%%%%%%%%%%%%%%%%%%%%%
% NotaBene (WIP)
\newenvironment{NB}[1]
    {
		\begin{description}
			\item[N.B. :]
			
			
    }
    { 
		\end{description}    
	}


%%%%%%%%%%%%%%%%%%%%%%%%%%%%%%%%%%%%%%%%%%%%%%%%%%%%%%%%%

%%%%%%%%%%%%%%%%%%%%%%%%%%%%%%%%%%%%%%%%%%%%%%%%%%%%%%%%%
%con opzione per il tratteggio
	\usepackage[framemethod=TikZ]{mdframed}  
	\mdfdefinestyle{dotted}{ tikzsetting={ 
		draw= white, 
		dash pattern = on 3pt off 3pt 
	}	}
   



 	\usepackage{amssymb} 

%%%%%%%%%%%%%%%%%%%%%%%%%%%%%%%%%%%%%%%%%%%%%%%%%%%%%%%%%
%Ipotesi con barra a sinistra

	\usetikzlibrary{calc}
%	\usepackage{fourier-orns}
	\tikzset{
  		box/.style={
      		rectangle,
      		draw=red,
      		fill=white,
      		scale=1,
      		overlay
    }		}

	\mdfdefinestyle{hypothesis}{%
 		hidealllines=true,leftline=true,
 		skipabove=12,skipbelow=12pt,
 		innertopmargin=0.4em,%
 		innerbottommargin=0.4em,%
 		innerrightmargin=0.7em,%
 		rightmargin=0.7em,%
 		innerleftmargin=1.7em,%
 		leftmargin=0.7em,%
 		middlelinewidth=.2em,%
 		linecolor=blue,%
 		firstextra={\path let \p1=(P), \p2=(O) in ($(\x2,0)+(0,\y1)$) 
                           node[box] {\textcolor{blue}{Hp:}};},%
 		secondextra={\path let \p1=(P), \p2=(O) in ($(\x2,0)+(0,\y1)$) 
                           node[box] {\textcolor{blue}{Hp:}};},%
 		middleextra={\path let \p1=(P), \p2=(O) in ($(\x2,0)+(0,\y1)$) 
                           node[box] {\textcolor{blue}{Hp:}};},%
 		singleextra={\path let \p1=(P), \p2=(O) in ($(\x2,0)+(0,\y1)$) 
                           node[box] {\textcolor{blue}{Hp:}};},%
	}

\newmdenv[style=hypothesis]{hypothesis}




%%%%%%%%%%%%%%%%%%%%%%%%%%%%%%%%%%%%%%%%%%%%%%%%%%%%%%%%%
%Tesi con barra a sinistra

	\mdfdefinestyle{thesis}{%
 		hidealllines=true,leftline=true,
 		skipabove=12,skipbelow=12pt,
 		innertopmargin=0.4em,%
 		innerbottommargin=0.4em,%
 		innerrightmargin=0.7em,%
 		rightmargin=0.7em,%
 		innerleftmargin=1.7em,%
 		leftmargin=0.7em,%
 		middlelinewidth=.2em,%
 		linecolor=blue,%
 		firstextra={\path let \p1=(P), \p2=(O) in ($(\x2,0)+(0,\y1)$) 
                           node[box] {\textcolor{blue}{Th:}};},%
 		secondextra={\path let \p1=(P), \p2=(O) in ($(\x2,0)+(0,\y1)$) 
                           node[box] {\textcolor{blue}{Th:}};},%
 		middleextra={\path let \p1=(P), \p2=(O) in ($(\x2,0)+(0,\y1)$) 
                           node[box] {\textcolor{blue}{Th:}};},%
 		singleextra={\path let \p1=(P), \p2=(O) in ($(\x2,0)+(0,\y1)$) 
                           node[box] {\textcolor{blue}{Th:}};},%
	}

\newmdenv[style=thesis]{thesis}



%%%%%%%%%%%%%%%%%%%%%%%%%%%%%%%%%%%%%%%%%%%%%%%%%%%%%%%%%
% Danger box (WIP).
% It encapsulates a piece of text that needs revision
% http://tex.stackexchange.com/questions/52023/mdframed-put-something-on-the-start-of-one-vertical-left-rule
%%%%%%%%%%%%%%%%%%%%%%%%%%%%%%%%%%%%%%%%%%%%%%%%%%%%%%%%%
\newenvironment{Warning}[1]
    {
		\begin{description}
			\item[!danger!]
    }
    { 
			\item[!danger!]
		\end{description}    
	}




\begin{document}


\section{Approvati (MD THEOREM di base)}
\lipsum[1]

\begin{lemma}[di coso]
	Enunciato...
Text Text Text Text Text Text $\sqrt{r}$ Text Text Text Text Text



\end{lemma}%

\begin{theorem}%
teorema
\end{theorem}%

\begin{corollary}%
corollary
\end{corollary}%

\begin{proposition}%
proposition
\end{proposition}%

\begin{proof}
ciao
\end{proof}
\proof
CiaoCiao
Ciao
\lipsum[3]
\endproof



\begin{definition}[Bu]
bbbb bbbbbb bbbbb bbbbb bbbbbbbbb bbbbb bbbb bbbbb bbbbbbbb bbbbbb bbbbbbbbbb bbbb bbbbb bbbbbb 
bbbbbbbb
\end{definition}


\section{WiP}

\begin{notationfix}
Text
\end{notationfix}

\begin{TAM}
Text
\end{TAM}

\begin{example}
Text
\end{example}

\begin{observation}
Text
\end{observation}

\begin{remark}
Text
\end{remark}


\section{List of all the sentences}

\subsection{Sentence Matematica}

\begin{itemize}
 \item Preambolo
 \item Definizione
 \item Osservazione
 \item Lemma
 \item Teorema
 \item Proposizione
 \item Corollario
 \item Fissare la notazione:
 	\\ Quando si da un nome particolare ad una piccola sottodefinizione oppure quando si attribuisce un simbolo a qualcosa
 \item Esempio 	
 \end{itemize}

\subsection{Sentence Fisica}
\begin{itemize}
 \item assioma di matematizzazione 
 	\\(e.g. spazio-tempo = varietà 4D)
 \item assioma come concetto primitivo 
 	\\(e.g. lo spazio fisico ha 3 dimensioni)
 \item assioma di reinterpretazione 
 	\\(e.g. non mi viene in mente ora... è qualcosa che dopo tanti calcoli tiro fuori e gli appioppo una qualche interpretazione fisica)
 \item giustificazione assioma 
 	\\L'assioma è qualcosa che non si può dimostrare ma tendenzialmente si può dare un argomento che cerca di giustificare il fatto che viene usato.. eg. l'omogeneità nello spazio-tempo presa per vera nei modelli FRW.
 \item Conferma sperimentale
 	\\ si mette qua dentro la reference a qualche espiremento che avvalora gli assiomi o altro
 \item Informazioni sul contesto.
 	\\ esempio, studio proteine, in questo contesto gli ordini di grandezza sono ...
 \item Postulato
 	\\Ci sono differenze tra postulato e assioma... assioma= verità autoevidente, postulato= affermazione assunta vera


\end{itemize}

\begin{Warning}
	Text
\end{Warning}

\end{document}